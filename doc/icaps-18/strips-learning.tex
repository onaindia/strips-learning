\def\year{2017}\relax
%File: formatting-instruction.tex
\documentclass[letterpaper]{article} %DO NOT CHANGE THIS
\usepackage{aaai17}  %Required
\usepackage{times}  %Required
\usepackage{helvet}  %Required
\usepackage{courier}  %Required
\usepackage{url}  %Required
\usepackage{graphicx}  %Required
\frenchspacing  %Required
\setlength{\pdfpagewidth}{8.5in}  %Required
\setlength{\pdfpageheight}{11in}  %Required
%PDF Info Is Required:
  \pdfinfo{}
\setcounter{secnumdepth}{0}

\usepackage{amsmath}
\usepackage{amssymb}
\usepackage{amsthm}
\usepackage{multirow}
\usepackage{tikz}
\usepackage{comment}

\usepackage{graphicx}
\usepackage{caption}
\usepackage{subcaption}
\usepackage{listings}
\usepackage{multicol}
\usepackage{arydshln}

\newcommand{\tup}[1]{{\langle #1 \rangle}}

\newcommand{\pre}{\mathsf{pre}}     % precondition
\newcommand{\del}{\mathsf{del}}     % effect
\newcommand{\add}{\mathsf{add}}     % effect
\newcommand{\eff}{\mathsf{eff}}     % effect
\newcommand{\cond}{\mathsf{cond}}   % conditional effect
\newcommand{\true}{\mathsf{true}}   % true
\newcommand{\false}{\mathsf{false}} % false
\newcommand{\PE}{\mathrm{PE}}     % precondition
\newcommand{\strips}{\textsc{Strips}}     % precondition

\newtheorem{theorem}{Theorem}
\newtheorem{lemma}[theorem]{Lemma}
\newtheorem{definition}[theorem]{Definition}


\begin{document}

\title{Learning STRIPS action models with classical planning}

\author{Diego Garc\'ia\and Sergio Jim\'enez\and Eva Onaindia\\
{\small Departamento de Sistemas Inform\'aticos y Computaci\'on}\\
{\small Universitat Polit\'ecnica de Val\'encia.}\\
{\small Camino de Vera s/n. 46022 Valencia, Spain}\\
{\small \{,serjice,onaindia\}@dsic.upv.es}}
 
\maketitle
\begin{abstract}
This paper presents a novel approach for learning classical planning action models from minimal input knowledge and exclusively using existing classical planners. The paper defines a classical planning compilation to learn action models from examples. The compilation is flexible to different degrees of available input knowledge; it accepts partially specified action models and learning examples can range from a set of plans with their corresponding initial and final states to only a set of initial and final states.  
\end{abstract}


\section{Introduction}
Off-the-shelf planners reason about action models that correctly and completely capture the possible world transitions~\cite{geffner:book:2013}. In addition planning action models allow to effectively address further tasks than plan synthesis, like plan/goal recognition~\cite{ramirez2010probabilistic}. Unfortunately building planning action models is complex, even for planning experts, limiting the potential of automated planning~\cite{kambhampati:modellite:AAAI2007}. 

Despite many Machine Learning (ML) techniques are able to compute models from examples~\cite{michalski2013machine} its application to learning planning action models is not straightforward. First, the inputs to ML algorithms usually are finite numeric vectors encoding objects features while, in planning, the input includes actions so is closer to the representation of procedures and behaviours rather than to object representation. Second, the traditional output of off-the-shelf ML techniques is a scalar value (an integer, in the case of classification tasks, or a real value, in the case of regression tasks). In the case of learning planning action models the output is not a scalar but a model of the possible state transitions. Last but not least, collecting {\em informative} examples for learning planning action models is challenging. Planning actions include preconditions that are only satisfied by specific sequences of actions, and often, with a low probability of being chosen by chance~\cite{fern2004learning}.

Learning classical planning action models is a well-studied problem with sophisticated algorithms, like ARMS~\cite{yang2007learning}, SLAF~\cite{amir:alearning:JAIR08} or LOCM~\cite{cresswell2013acquiring} that do not require full knowledge of all the states traversed by the example plans. Motivated by recent advances on learning generative models with classical planning~\cite{segovia2017generating} and on the effective exploration of planning state spaces~\cite{geffner:novelty:IJCAI17}, this paper introduces an innovative approach for learning classical planning action models that offers the following contribution: An inductive learning algorithm that minimizes the required input knowledge, that is flexible to different degrees of available input knowledge, and that can be defined as a classical planning compilation.


\section{Background}
This section defines the planning models used on this work.

\subsection{Classical Planning}
We use $F$ to denote the set of {\em fluents} (propositional variables) describing a state. A {\em literal} $l$ is a valuation of a fluent $f\in F$, i.e.~$l=f$ or $l=\neg f$. A set of literals $L$ represents a partial assignment of values to fluents (WLOG we assume that $L$ does not assign conflicting values to any fluent). We use $\mathcal{L}(F)$ to denote the set of all literal sets on $F$, i.e.~all partial assignments of values to fluents. A {\em state} $s$ is then a total assignment of values to fluents, i.e. $|s|=|F|$, so the size of the state space $2^{|F|}$. 

A {\em classical planning frame} is a tuple $\Phi=\tup{F,A}$, where $F$ is a set of fluents and $A$ is a set of actions. Each action $a\in A$ has a set of literals $\pre(a)\in\mathcal{L}(F)$, called {\em preconditions}, a set of positive effects $\add(a)\in\mathcal{L}(F)$, and a set of negative effects $\del(a)\in\mathcal{L}(F)$. An action $a\in A$ is applicable in state $s$ iff $\pre(a)\subseteq s$, and the result of applying $a$ in $s$ is a new state $\theta(s,a)=(s\setminus \neg\del(a))\cup\add(a)$.

A {\em classical planning problem} is a tuple $P=\tup{F,A,I,G}$, where $I$ is an initial state and $G\in\mathcal{L}(F)$ is a goal condition. A {\em plan} for $P$ is an action sequence $\pi=\tup{a_1, \ldots, a_n}$ that induces a state sequence $\tup{s_0, s_1, \ldots, s_n}$ such that $s_0=I$ and, for each {\small $1\leq i\leq n$}, $a_i$ is applicable in $s_{i-1}$ and generates the successor state $s_i=\theta(s_{i-1},a_i)$. The plan $\pi$ {\em solves} $P$ if and only if $G\subseteq s_n$, i.e.~if the goal condition is satisfied following the application of $\pi$ in $I$.

In this work we assume that fluents $F$ are instantiated from predicates, as in PDDL~\cite{fox2003pddl2}. There exists a set of predicates $\Psi$, each $p\in\Psi$ with an argument list of arity $ar(p)$. Given a set of objects $\Omega$, the set of fluents $F$ is then induced by assigning objects in $\Omega$ to the arguments of predicates in $\Psi$, i.e.~$F=\{p(\omega):p\in\Psi,\omega\in\Omega^{ar(p)}\}$ s.t. $\Omega^n$ is the $n$-th Cartesian power of $\Omega$.

Likewise we assume that actions in $A$ are instantiated from operator schema. Let $\Omega_v=\{v_1,\ldots,v_v\}$, $\Omega_v\cap\Omega=\emptyset$ be a new set of objects representing variable names defined by the action with the maximum arity in a planning frame. For instance, in the blocksworld $\Omega_v=\{v_1,v_2\}$ since operators $stack$ (see Figure~\ref{fig:stack}) and $unstack$ have two parameters. Let us define a new set of fluents $F_{v}$ that results instantiating $\Psi$ but using only the objects in $\Omega_v$, e.g. for the blocksworld example $F_v=\{handempty, holding(v_1), holding(v_2), clear(v_1), \\clear(v_2), ontable(v_1), ontable(v_2), on(v_1,v_1),on(v_1,v_2),\\on(v_2,v_1),on(v_2,v_2)\}$.

\begin{figure}[hbt]
\begin{footnotesize}
\begin{verbatim}
  (:action stack
   :parameters (?x1 ?x2)
   :precondition (and (holding ?x1) 
                      (clear ?x2))
   :effect (and (not (holding ?x1))
                (not (clear ?x2))
                (clear ?x1)
                (handempty)
                (on ?x1 ?x2)))
\end{verbatim}
\end{footnotesize}
 \caption{\small Example of the {\em stack} planning operator schema from the blocksworld as represented in PDDL.}
\label{fig:stack}
\end{figure}

We are ready to define an operator schema $\xi\in \Xi$ as a tuple $\tup{head(\xi),pre(\xi),add(\xi),del(\xi)}$ where:
\begin{itemize}
\item $head(\xi)$, represents the operator {\em header} defined by a pair $head(\xi)=(name(\xi),pars(\xi))$ built using an action name and a list of variables, $pars(\xi)\in\Omega_v^{ar(\xi)}$. The blocksworld only requires these four operator headers $\{pickup(v_1), putdown(v_1), stack(v_1,v_2), unstack(v_1,v_2)\}$.
\item The preconditions $pre(\xi)\subseteq F_v$, the positive effects $add(\xi)\subseteq F_v$, and the negative effects $del(\xi)\subseteq F_v$ such that, $del(\xi)\subseteq pre(\xi)$, $del(\xi)\cap add(\xi)=\emptyset$ and $pre(\xi)\cap add(\xi)=\emptyset$.
\end{itemize}


\subsection{Classical Planning with Conditional Effects}
Conditional effects make it possible to repeatedly refer to the same action while their precise effects depend on the current state. Now an action $a\in A$ has a set of literals $\pre(a)\in\mathcal{L}(F)$ called the {\em precondition} and a set of conditional effects $\cond(a)$. Each conditional effect $C\rhd E\in\cond(a)$ is composed of two sets of literals $C\in\mathcal{L}(F)$ (the condition) and $E\in\mathcal{L}(F)$ (the effect).

An action $a\in A$ is applicable in state $s$ if and only if $\pre(a)\subseteq s$, and the resulting set of {\em triggered effects} is
\[
\eff(s,a)=\bigcup_{C\rhd E\in\cond(a),C\subseteq s} E,
\]
i.e.~effects whose conditions hold in $s$. The result of applying $a$ in $s$ is a new state $\theta(s,a)=(s\setminus \neg\eff(s,a))\cup\eff(s,a)$.


\section{Learning classical planning action models}
Learning classical planning action models from fully available input knowledge, i.e. a set of plans where the {\em pre-} and {\em post-states} of every action in a plan are available, is straightforward: The operators schema are derived lifting the literals that change between the pre and post-state of the corresponding action executions. Preconditions are derived lifting the minimal set of literals appearing in all the pre-states that correspond to the same operator.

This section formalizes more challenging tasks, for learning classical planning action model, where less input knowledge is available. Formalization is done according to the available input knowledge.

\subsection{Learning from labeled plans}
This learning task is formalized as $\Lambda=\tup{\Psi,\Pi,\Sigma}$: 
\begin{itemize}
\item $\Psi$ the set of predicates that define the abstract state space of a given planning domain. This set includes the predicates for defining the headers of the operators schema.
\item $\Pi=\{\pi_1,\ldots,\pi_t\}$ is the given set of example plans,
\item $\Sigma=\{\sigma_1,,\ldots,\sigma_t\}$ is a set of labels s.t. each plan $\pi_i$, {\small $1\leq i\leq t$}, has an associated label $\sigma_i=(s_i,s_i')$ where $s_i'$ is the state resulting from executing $\pi_i$ starting from the state $s_i$. 
\end{itemize}

A solution to the learning task $\Lambda$ is a set of operator schema $\Xi$, with one schema for each operator header, compliant with the predicates in $\Psi$, the example plans $\Pi$, and their labels $\Sigma$.

\subsection{Learning from states}
Here we reduce the amount of input knowledge provided to the previous learning task. Now each $\Pi=\{\pi_1,\ldots,\pi_t\}$ does not contain a set of plans but the number of actions of each plan, $\Pi'=\{|\pi_1|,\ldots,|\pi_t|\}$ and redefine the learning task as $\Lambda'=\tup{\Psi,\Pi',\Sigma}$. While the previous task can correspond to watching an agent acting in the world, this learnig tasks can be understood as watching only the results of its actions executions knowing the number of diferent actions performed by the agent.

We can go a step further and redefine the learning task as $\Lambda''=\tup{\Psi,\Sigma}$. This learnig tasks corresponds to watching only the results of the plan executions. Now a solution to the $\Lambda'$ learning task is a set of operator schema $\Xi$ that is compliant only with the predicates in $\Psi$, and the given set of initial and final states.


\section{Learning action models from example states using a classical planner}
Our approach for addressing $\Lambda''=\tup{\Psi,\Sigma}$ is compiling this learning task into a classical planning task $P_{\Lambda}$. The intuition behind the compilation is that a solution to $P_{\Lambda}$ is a sequence of actions that first, programs the action model (i.e. the $pre(\xi)$, $del(\xi)$ and $add(\xi)$ for each $\xi\in\Xi$) and then, validates the programmed action model in the given set of labels $\Sigma$, one after the other.

To formalize the compilation we define $t$ classical planning instances $P_1=\tup{F,A,I_1,G_1},\ldots,P_t=\tup{F,A,I_t,G_t}$, that belong to the same planning frame (share the same fluents and actions and differ only in the initial state and goals). The set of fluents $F$ is built instantiating the predicates in $\Psi$ with the objects in $\Omega=\{o|o\in s_i\cup s_i', {\small 1\leq i\leq t}\}$, the set of objects that appear in the fluents $f\in F$ used to define the states in $\Sigma$. The set of actions is empty $A=\emptyset$. Finally the initial state $I_i$, {\small $1\leq i\leq t$}, is given by the state $s_i\in \sigma_i$ while goals $G_i$ are defined by the state $s_i'\in \sigma_i$. 

Now we are ready to define the compilation for learning action models using an off-the-shelf classical planner. Given a learning task $\Lambda'=\tup{\Psi,\Sigma}$ the compilation outputs a classical planning task $P_{\Lambda}=\tup{F_{\Lambda},A_{\Lambda},I_{\Lambda},G_{\Lambda}}$ where:
\begin{itemize}
\item $F_{\Lambda}$ extends $F$ with:
\begin{itemize}
\item Fluents representing the programmed action model: $pre_f(\xi)$, $del_f(\xi)$ and $add_f(\xi)$ for every $f\in F_v$ and $\xi \in \Xi$.
\item Fluents $\{test_i\}_{1\leq i\leq t}$, indicating the example where the programmed model is currently being validated.
\item Fluents $prog1$, $prog2$ and $exec$ indicating whether the solution is programming the preconditions of the action schema, the effects, or it started validating the programmed action models.
\end{itemize}
\item $I_{\Lambda}$, contains the fluents from $F$ that encode the initial state $s_1\in P_1$, every fluent $pre_f(\xi)\in F_{\Lambda}$ and $prog1$. 
\item $G_{\Lambda}=\{test_i\}$,{\small $1\leq i\leq t$}, indicates that the programmed action model is validated in all the examples in $\Sigma$.
\item $A_{\Lambda}$ replaces the actions in $A$ with actions of three types:
\begin{enumerate}
\item Actions for programming:

\begin{itemize}
\item {\em Precondition} $f\in F_v$ in the action schema $\xi\in\Xi$:
\begin{small}
\begin{align*}
\hspace*{7pt}\pre(\mathsf{programPre_{f,\xi}})=&\{pre_{f}(\xi),\neg prog2,\neg exec\},\\    
\cond(\mathsf{programPre_{f,\xi}})=&\{\emptyset\}\rhd\{\neg pre_{f}(\xi)\}.
\end{align*}
\end{small}

\item {\em Negative effect} $f\in F_v$ in the action schema $\xi\in\Xi$:
\begin{small}
\begin{align*}
\hspace*{7pt}\pre(\mathsf{programDel_{f,\xi}})=&\{pre_{f}(\xi),\neg del_{f}(\xi),\neg add_{f}(\xi),\neg exec\},\\                                                   
\cond(\mathsf{programDel_{f,\xi}})=&\{\emptyset\}\rhd\{del_{f}(\xi),\\
&\{\emptyset\}\rhd\{prog2\}.
\end{align*}
\end{small}

\item {\em Positive effect} $f\in F_v$ in the action schema $\xi\in\Xi$:
\begin{small}
\begin{align*}
\hspace*{7pt}\pre(\mathsf{programAdd_{f,\xi}})=&\{\neg pre_{f}(\xi)),\neg del_{f}(\xi)),\neg add_{f}(\xi),\neg exec\},\\                                                   
\cond(\mathsf{programAdd_{f,\xi}})=&\{\emptyset\}\rhd\{add_{f}(\xi),\\
&\{\emptyset\}\rhd\{prog2\}.
\end{align*}
\end{small}
\end{itemize}

\item Actions for applying an already programmed operator schema $\xi\in\Xi$ bounding it with objects $\omega\subseteq\Omega^{ar(\xi)}$
\begin{small}
\begin{align*}
\hspace*{7pt}\pre(\mathsf{apply_{\xi,\omega}})=&\{pre_{f}(\xi)\implies p(\omega)\}_{\forall p\in\Psi,f=p(pars(\xi))},\\
\cond(\mathsf{apply_{\xi,\upsilon,\upsilon'}})=&\{del_{f}(\xi)\}\rhd\{\neg p(\omega)\}_{\forall p\in\Psi,f=p(pars(\xi))},\\
&\{add_{f}(\xi)\}\rhd\{p(\omega)\}_{\forall p\in\Psi,f=p(pars(\xi))},\\
&\{\emptyset\}\rhd\{exec\}.
\end{align*}
\end{small}

For instance, these actions define that if an operator is programmed with precondition $holding(v_1)\in F_v$ it implies that $holding(block_1)\in F$ has to be true in the current state if the operator binds variable object $v_1\in\Omega_v$ with object $block_1\in\Omega_v$. The operator binding is done implicitly, i.e. variables in $pars(\xi)$ are bound to the objects in $\omega$ appearing in the same position. 

\item Actions for changing the active example where the action model is currently being validated.
\begin{small}
\begin{align*}
\hspace*{7pt}\pre(\mathsf{validate_{i}})=&G_i\cup\{test_j\}_{j\in 1\leq j<i}\cup\{exec\},\\
\cond(\mathsf{validate_{i}})=&\{\emptyset\}\rhd\{test_i\}.
\end{align*}
\end{small}
\end{enumerate}
\end{itemize}


\begin{lemma}
Any classical plan $\pi$ that solves $P_{\Lambda}$ induces a valid action model that solves the learning task $\Lambda$.
\end{lemma}

\begin{proof}[Proof sketch]
Once an operator schema is programmed it cannot be modified and can only be applied. The only way of achieving a {\em test} fluent is by applying a sequence of programmed operator schema until achieving the goal state defined by its associated label starting from the initial state of the corresponding label. If this is done for all the labels (all the input examples) it means that the programmed action model is compliant with the learning input knowledge and hence, it is a solution to the action model learning task.
\end{proof}

The compilation is valid for partially specified action models since known preconditions and effects (fluents $pre_f(\xi)$, $del_f(\xi)$ and $add_f(\xi)$) can be part of the initial state $I_{\Lambda}$. The approach allows also transfer learning where the action model generated for a given sub-task is encoded as {\em already programmed} for learning new action models in more challenging tasks.

\section{Learning action models from example plans using a classical planner}
The compilation can be extended to the learning scenario defined by $\Lambda$ and $\Lambda'$ in which a set of plans $\Pi$ (or only its lengths in the case of $\Lambda'$) is available. Each plan $\pi_i\in \Pi$, {\small $1\leq i\leq t$}, is a solution to the corresponding classical planning instance $P_i=\tup{F,A,I_i,G_i}$ defined above. The compilation extensions are: 
\begin{itemize}
\item $F_{\Lambda}$ includes the new set of fluents $F_{\Pi}=\{plan(name(\xi),j,\Omega^{ar(\xi)})\}$ for encoding the $j$ steps of the {\small $1\leq i\leq t$} plans in $\Pi$ with $F_{\Pi_i}\subseteq F_{\Pi}$ the fluents encoding the plan corresponding to the $i^{th}$ example (only for the $\Lambda$ case). In addition fluents $at_j$ and $next_{j,j_2}$, {\small $1\leq j<j2\leq n$}, represent the plan step where the programmed action model is validated ($n$ is the max length of a plan in $\Pi$).
\item $I_{\Lambda}$ is extended with the fluents from $F_{\Pi_1}$ that encode the plan $\pi_1\in \Pi$ for solving $P_1$, and the fluents $at_1$ and $\{next_{j,j_2}\}$, {\small $1\leq j<j2\leq n$}, for indicating the plan step where to start validating the programmed action model. Goals $G_{\Lambda}$ are like in the original compilation.
\item With respect to the actions in $A_{\Lambda}$,
\begin{enumerate}
\item The actions for programming the preconditions/effects of a given operator schema are the same.
\item The actions for applying an operator schema have an extra precondition $f\in F_{\Pi_i}$ that encodes the current plan step and extra conditional effect $\{at_{j}\}\rhd\{\neg at_{j},at_{j+1}\}_{\forall j\in [1,n]}$ for advancing the plan step.
\item The actions for changing the active test have an extra precondition, $at_{|\Pi_i|}$, to indicate that we simulated the full current plan $\Pi_i$ and extra conditional effects to load the next plan $\Pi_{i+1}$ where to validate the programmed action model:
\begin{small}
\begin{align*}
&\{f\}\rhd\{\neg f\}_{f\in F_{\Pi_i}},\\
&\{\emptyset\}\rhd\{f\}_{f\in F_{\Pi_i+1}},\\
&\{\emptyset\}\rhd\{\neg at_{|\pi_i|},at_1\}.
\end{align*}
\end{small}
\end{enumerate}
\end{itemize}


\section{Evaluation}

\subsection{Learning action models from example plans}
\begin{table*}[!hbt]
\begin{footnotesize}
\begin{tabular}{l||c|c|c||c|c|c||c|c|c}
handpicked/blocks 0.00 0.00 0.00 0.00 0.00 0.00
handpicked/gripper 17.78 13.70 0.00 0.00 0.00 0.00
handpicked/miconic 54.17 27.32 0.00 0.00 0.00 0.00
handpicked/visitall 0.00 0.00 0.00 0.00 0.00 0.00

\end{tabular}
\end{footnotesize}
\caption{\small Mean error and standard deviation of the learned models when using hand-picked examples and examples collected using the classical planner Fast-Downward.}
  \label{tab:eplans}
\end{table*}

The performance of our learning approach is evaluated for different degrees of available input knowledge and using different sources for collecting this input knowledge. In all the cases we assess the performance of our learning approach using the cardinality of the {\em symmetric difference} sets that are computed between the set of preconditions, del and add effects (1), in the learned model and (2), in the actual models. In all the experiments the compilation is solved using the SAT-based classical planner {\sc Madagascar}~\cite{rintanen2014madagascar}.

Table~\ref{tab:eplans} shows the mean error and standard deviation of the learned models with respect to the actual action models when (1) using {\em hand-picked} examples, (2) examples collected using the classical {\em planner} {\sc Fast-Downward}~\cite{helmert2006fast} and (3) examples collected {\em randomly}. The standard deviation provides a measure of how this error is distributed among the different operators in the domain. If this deviation is 0 it means that is equally distributed in all the domain operators.


\subsection{Learning action models from example states}
TBD.


\section{Related work}

%The  LIVE  system  (Shen  and  Simon,  1989)  was  an  extension  of  the  General  Problem  Solver  (GPS) framework (Ernst and Newell, 1969) with a learning component. LIVE alternated problem solving with model  learning  to  automatically  define  operators.  The  decision  about  when  to  alternate  depended  onsurprises,  that  is  situations  where  an  action  effects  violated  its  predicted  model.  EXPO  (Gil,  1992) generated plans with the  PRODIGY  system (Minton, 1988), monitored the plans execution, detected differences in the predicted and the observed states and constructed a set of specific hypotheses to fix those differences. Then the  EXPO  filtered the hypotheses heuristically.  OBSERVER  (Wang, 1994) learned operators by monitoring expert agents and applying the version spaces algorithm (Mitchell, 1997) to the observations. When the system already had an operator representation, the preconditions were updated by removing facts that were not present in the new observation’s pre-state; the effects were augmented by adding facts that were in the observation’s delta-state. All of these early works were based on direct liftings of the observed states. They also benefit from experience beyond simple interaction with the environment such as exploratory plans or external teachers, but  none  provided  a  theoretical  justification  for  this  second  source  of  knowledge.  The  work  recently reported in (Walsh and Littman, 2008) succeeds in bounding the number of interactions the learner must complete to learn the preconditions and effects of a STRIPS action model. This work shows that learning STRIPS  operators  from  pure  interaction  with  the  environment,  can  require  an  exponential  number  of samples, but that limiting the size of the precondition lists enable sample-efficient learning (polynomial in the number of actions and predicates of the domain). The work also proves that efficient learning is also possible without this limit if an agent has access to an external teacher that can provide solution traces on demand.

%Others systems have tried to learn more expressive action models for deterministic planning in fully observable environments. Examples would include the learning of conditional costs for AP actions (Jess Lanchas and Borrajo, 2007) or the learning of conditional effects with quantifiers (Zhuo et al., 2008).

%In addition action model learning has been studied in domains where there is partial state observability. ARMS uses the same kind od learning examples but assumes the exampls are given and proceeds in two phases. In the first phase, ARMS extracts frequent action sets from plans that share a common set of parameters. ARMS also finds some frequent literal-action pairs with the help of the initial state and the goal state that provide an initial guess on the actions preconditions, and effects. In the second phase, ARMS uses the frequent action sets and literal-action pairs to define a set of weighted constraints that must hold in order to make the plans correct. Then, ARMS solves the resulting weighted MAX-SAT problem and produces action models from the solution of the SAT problem. This process iterates until all actions are modelled. For a complex planning domain that involves hundreds of literals and actions, the corresponding weighted MAX-SAT representation is likely to be too large to be solved efficiently as the number of clauses can reach up to tens of thousands. For that reason ARMS implements a hill-climbing method that models the actions approximately. Consequently, the ARMS output is a model which may be inconsistent with the examples.

%(Amir and Chang, 2008) introduced an algorithm that tractably generates all the STRIPS-like models that  could  have  lead  to  a  set  of  observations.  Given  a  formula  representing  the  initial  belief  state,  a sequence of executed actions and the corresponding observed states(where partial observations of states are given), it builds a complete explanation of observations by models of actions through a Conjunctive Normal Form (CNF) formula. By linking the possible states of fluents to the effect propositions in the action models, the complexity of the CNF encoding can be controlled to find  exact  solutions  efficiently  in  some  circumstances.  The  learning  algorithm  updates  the  formula  of the belief state with every action and observation in the sequence. This update makes sure that the new formula represents all the transition relations consistent with the actions and observations. The formula returned at the end includes all consistent models, which can then be retrieved with additional processing. Unlike the previous approaches, the one described in (Mour ao et al., 2008) deals with both missing and noisy predicates in the observations. For each action in a given domain, they use kernel perceptrons to learn predictions of the domain properties that change because of the action execution. LOCM (Cresswellet al., 2009) induces action schemas without being provided with any information about initial, goal or intermediate state descriptions for the example action sequences. LOCM receives descriptions of plans or plan fragments, uses them to create states machines for the different domain objects and extracts the action schemas from these state machines.





\section{Conclusions}
This paper presents a novel approach for learning classical planning action models from minimal input knowledge and using exclusively existing classical planners. Learning action models from examples allows the reformulation of a domain theory. An interesting research direction is the study of domain reformulation using features that allow more compact solutions like the {\em reachable} or {\em movable} features in the Sokoban domain.

\bibliographystyle{aaai}
\bibliography{strips-learning}
\end{document}
